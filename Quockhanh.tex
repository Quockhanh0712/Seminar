\documentclass[a4paper,12pt]{article}
\usepackage{graphicx}
\usepackage{amsmath, amssymb}
\usepackage{booktabs}
\usepackage{hyperref}
\usepackage{enumitem}

\title{Phương pháp thực hiện và đánh giá}
\author{}
\date{}

\begin{document}
\maketitle

\section{Mục tiêu}
Nghiên cứu về phương pháp thực hiện và đánh giá, thực hiện lại mô hình.

\section{Tổng Quan}
\subsection{Key Points}
\begin{itemize}
    \item Nghiên cứu cho thấy bài báo tập trung vào việc tạo dữ liệu tài chính tổng hợp để phân tích cảm xúc, giải quyết vấn đề bảo mật dữ liệu và thiếu dữ liệu được gắn nhãn.
    \item Phương pháp \textit{Reinforcement Prompting} sử dụng mạng lưới chính sách và mô hình ngôn ngữ lớn (LLM) để tạo dữ liệu, với hiệu quả được đánh giá qua các chỉ số như độ chính xác và sự tương đồng phong cách.
    \item Kết quả cho thấy dữ liệu tổng hợp có thể cạnh tranh với dữ liệu thực, nhưng vẫn tồn tại thách thức như sự phụ thuộc vào dữ liệu thực và khả năng tổng quát hóa.
    \item Các vấn đề mở bao gồm cải thiện hiệu suất trên tập dữ liệu đa dạng và giảm phụ thuộc vào tài nguyên tính toán.
\end{itemize}

\section{Giới thiệu về bài báo}
Bài báo \textit{Reinforcement Prompting for Financial Synthetic Data Generation} khám phá cách tạo dữ liệu tài chính tổng hợp để hỗ trợ phân tích cảm xúc, một lĩnh vực quan trọng trong tài chính. Nghiên cứu này giải quyết hai vấn đề chính: bảo mật dữ liệu và sự khan hiếm dữ liệu được gắn nhãn, vốn là rào cản lớn trong việc huấn luyện mô hình học máy.

\section{Bài toán}
Bài toán chính là tạo ra dữ liệu tài chính tổng hợp chất lượng cao để sử dụng trong phân tích cảm xúc, trong khi vẫn đảm bảo bảo mật dữ liệu và giảm sự phụ thuộc vào dữ liệu được gắn nhãn.

\section{Ý nghĩa ứng dụng}
Việc tạo dữ liệu tổng hợp có ý nghĩa lớn trong lĩnh vực tài chính, đặc biệt khi dữ liệu thực thường bị hạn chế do quy định bảo mật và chi phí gắn nhãn.

\section{Các tiếp cận giải quyết}
Phương pháp \textit{Reinforcement Prompting} được đề xuất như một giải pháp, sử dụng học tăng cường và LLM để tạo dữ liệu tổng hợp.
\begin{itemize}
    \item \textbf{Selector (Mạng lưới chính sách):} Chọn các từ khóa từ một từ điển thuật ngữ tài chính.
    \item \textbf{Executor (LLM):} Sử dụng các từ khóa này để tạo câu văn tài chính tổng hợp.
    \item \textbf{Học tăng cường:} Mạng lưới chính sách được huấn luyện bằng thuật toán REINFORCE.
\end{itemize}

\section{Phương pháp đánh giá}
Hiệu quả của phương pháp được đánh giá qua các thí nghiệm, so sánh mô hình huấn luyện trên dữ liệu tổng hợp với mô hình trên dữ liệu thực.

\subsection{Dữ liệu và mô hình}
\begin{itemize}
    \item Sử dụng tập dữ liệu Financial PhraseBank, Twitter Financial và Fin-News Financial.
    \item Mô hình BERT (FinBERT) được huấn luyện trên cả dữ liệu tổng hợp (BERT-Syn) và thực (BERT-Real).
\end{itemize}

\subsection{Kết quả thí nghiệm}
\begin{itemize}
    \item Trên Twitter Financial, BERT-Syn đạt độ chính xác cao nhất 0,67 so với BERT-Real đạt 0,64.
    \item Trên Fin-News Financial, BERT-Real tốt hơn với độ chính xác cao nhất 0,90 so với BERT-Syn đạt 0,84.
\end{itemize}

\subsection{Đánh giá con người}
Một nghiên cứu trường hợp với 3 chuyên gia tài chính đánh giá 1.000 câu từ cả hai tập dữ liệu:
\begin{table}[h]
    \centering
    \begin{tabular}{lcc}
        \toprule
        \textbf{Tiêu chí} & \textbf{Dữ liệu tổng hợp} & \textbf{Dữ liệu thực} \\
        \midrule
        Sự mạch lạc ngữ nghĩa & 4,3 & 4,5 \\
        Tính liên quan lĩnh vực & 4,1 & 4,2 \\
        \bottomrule
    \end{tabular}
    \caption{Bảng kết quả đánh giá con người}
\end{table}

\section{Các vấn đề mở}
\begin{itemize}
    \item Biến thiên do kích thước mẫu nhỏ.
    \item Phụ thuộc vào dữ liệu thực.
    \item Khả năng tổng quát hóa.
    \item Thiết kế hàm phần thưởng có thể mang tính chủ quan.
    \item Yêu cầu nhiều tài nguyên tính toán.
    \item Tính chuyên biệt lĩnh vực.
\end{itemize}

\section{Hướng nghiên cứu tương lai}
\begin{itemize}
    \item Cải thiện khả năng tổng quát hóa.
    \item Tối ưu hóa hàm phần thưởng.
    \item Giảm nhu cầu tài nguyên tính toán.
    \item Áp dụng cho các lĩnh vực khác như y tế, pháp lý.
    \item Kết hợp với học liên kết (federated learning).
    \item Thực hiện nghiên cứu dài hạn và kiểm chứng thực tế.
    \item Khám phá mô hình lai.
\end{itemize}

\section{Kết luận}
Bài báo cung cấp một phương pháp sáng tạo để tạo dữ liệu tài chính tổng hợp, giải quyết các vấn đề bảo mật và khan hiếm dữ liệu. Dù có những hạn chế, kết quả cho thấy tiềm năng lớn trong phân tích cảm xúc tài chính.

\section{Tài liệu tham khảo}
\begin{itemize}
    \item \textit{Reinforcement Prompting for Financial Synthetic Data Generation} - Full Paper.
\end{itemize}

\end{document}
